%%
%% This is file `sample-manuscript.tex',
%% generated with the docstrip utility.
%%
%% The original source files were:
%%
%% samples.dtx  (with options: `all,proceedings,bibtex,manuscript')
%% 
%% IMPORTANT NOTICE:
%% 
%% For the copyright see the source file.
%% 
%% Any modified versions of this file must be renamed
%% with new filenames distinct from sample-manuscript.tex.
%% 
%% For distribution of the original source see the terms
%% for copying and modification in the file samples.dtx.
%% 
%% This generated file may be distributed as long as the
%% original source files, as listed above, are part of the
%% same distribution. (The sources need not necessarily be
%% in the same archive or directory.)
%%
%%
%% Commands for TeXCount
%TC:macro \cite [option:text,text]
%TC:macro \citep [option:text,text]
%TC:macro \citet [option:text,text]
%TC:envir table 0 1
%TC:envir table* 0 1
%TC:envir tabular [ignore] word
%TC:envir displaymath 0 word
%TC:envir math 0 word
%TC:envir comment 0 0
%%
%% The first command in your LaTeX source must be the \documentclass
%% command.
%%
%% For submission and review of your manuscript please change the
%% command to \documentclass[manuscript, screen, review]{acmart}.
%%
%% When submitting camera ready or to TAPS, please change the command
%% to \documentclass[sigconf]{acmart} or whichever template is required
%% for your publication.
%%
%%
\documentclass[manuscript,screen,review]{acmart}
%%
%% \BibTeX command to typeset BibTeX logo in the docs
\AtBeginDocument{%
  \providecommand\BibTeX{{%
    Bib\TeX}}}

%% Rights management information.  This information is sent to you
%% when you complete the rights form.  These commands have SAMPLE
%% values in them; it is your responsibility as an author to replace
%% the commands and values with those provided to you when you
%% complete the rights form.
\setcopyright{acmlicensed}
\copyrightyear{2026}
\acmYear{2026}
\acmDOI{XXXXXXX.XXXXXXX}
%% These commands are for a PROCEEDINGS abstract or paper.
%\acmConference[Conference acronym 'XX]{Make sure to enter the correct
%  conference title from your rights confirmation email}{June 03--05,
%  2018}{Woodstock, NY}
%%
%%  Uncomment \acmBooktitle if the title of the proceedings is different
%%  from ``Proceedings of ...''!
%%
 \acmBooktitle{AI Is Here To Stay: Misinformation and Human-Centric Models Between Risks
and Opportunities,
   January 16, 2026, Padua, IT}
 \acmISBN{978-1-4503-XXXX-X/2026/01}


%%
%% Submission ID.
%% Use this when submitting an article to a sponsored event. You'll
%% receive a unique submission ID from the organizers
%% of the event, and this ID should be used as the parameter to this command.
%%\acmSubmissionID{123-A56-BU3}

%%
%% For managing citations, it is recommended to use bibliography
%% files in BibTeX format.
%%
%% You can then either use BibTeX with the ACM-Reference-Format style,
%% or BibLaTeX with the acmnumeric or acmauthoryear sytles, that include
%% support for advanced citation of software artefact from the
%% biblatex-software package, also separately available on CTAN.
%%
%% Look at the sample-*-biblatex.tex files for templates showcasing
%% the biblatex styles.
%%

%%
%% The majority of ACM publications use numbered citations and
%% references.  The command \citestyle{authoryear} switches to the
%% "author year" style.
%%
%% If you are preparing content for an event
%% sponsored by ACM SIGGRAPH, you must use the "author year" style of
%% citations and references.
%% Uncommenting
%% the next command will enable that style.
%%\citestyle{acmauthoryear}

\usepackage{listings}
\usepackage{xcolor}

\definecolor{codegreen}{rgb}{0,0.6,0}
\definecolor{codegray}{rgb}{0.5,0.5,0.5}
\definecolor{codepurple}{rgb}{0.58,0,0.82}
\definecolor{backcolour}{rgb}{0.96,0.96,0.95}

\definecolor{commentcolor}{HTML}{05700E}

\lstdefinestyle{mystyle}{ 
    backgroundcolor=\color{backcolour},
    commentstyle=\color{commentcolor},
    keywordstyle=\color{blue}\bfseries,
    numberstyle=\tiny\color{codegray},
    stringstyle=\color{codepurple},
    basicstyle=\ttfamily\footnotesize,
    breakatwhitespace=false,         
    breaklines=false,                 
    captionpos=b,                    
    keepspaces=true,                 
    numbers=left,                    
    numbersep=5pt,                  
    showspaces=false,                
    showstringspaces=false,
    showtabs=false,                  
    tabsize=2,
    frame=lines,
    captionpos = t,
}

\lstset{style=mystyle}

\usepackage{fontawesome5}
\usepackage{subcaption}


\definecolor{changes}{HTML}{A42A04}

%%
%% end of the preamble, start of the body of the document source.
\begin{document}

%%
%% The "title" command has an optional parameter,
%% allowing the author to define a "short title" to be used in page headers.
\title{AI Is Here To Stay: Misinformation and Human-Centric Models Between Risks and Opportunities}

%%
%% The "author" command and its associated commands are used to define
%% the authors and their affiliations.
%% Of note is the shared affiliation of the first two authors, and the
%% "authornote" and "authornotemark" commands
%% used to denote shared contribution to the research.
\author{Marco Bellò}
\email{marco.bello.3@studenti.unipd.it}

%\authornotemark[1]

\affiliation{%
  \institution{University of Padua}
  \city{Padua}
  \country{Italy}
}

%%
%% By default, the full list of authors will be used in the page
%% headers. Often, this list is too long, and will overlap
%% other information printed in the page headers. This command allows
%% the author to define a more concise list
%% of authors' names for this purpose.
\renewcommand{\shortauthors}{Bellò}

%%%%
%%%% The abstract is a short summary of the work to be presented in the
%%%% article.
\begin{abstract}
For more than half a century, artificial intelligence remained a matter of speculation for novelists and philosophers. This changed in 2017 due to the introduction of the transformer architecture, spearheading the AI boom whose effects are still scarcely understood.
This paper explores potential risks and use cases of this technology, with a "human-centric" cut: 
the first half showcases generation and detection of fake news and deepfakes, as well as how much LLMs can influence people's opinions and beliefs. The focus then shifts to the AIs themselves, about their own biases and their susceptibility to external stimuli, concluding with a quick overview on the state of research about AIs for positive social impact.
\end{abstract}
%%
%%
%%%%
%%%% The code below is generated by the tool at http://dl.acm.org/ccs.cfm.
%%%% Please copy and paste the code instead of the example below.
%%%%
\begin{CCSXML}
<ccs2012>
   <concept>
       <concept_id>10002944.10011122.10002945</concept_id>
       <concept_desc>General and reference~Surveys and overviews</concept_desc>
       <concept_significance>500</concept_significance>
       </concept>
   <concept>
       <concept_id>10010147.10010178.10010179</concept_id>
       <concept_desc>Computing methodologies~Natural language processing</concept_desc>
       <concept_significance>500</concept_significance>
       </concept>
   <concept>
       <concept_id>10010147.10010178.10010187.10010198</concept_id>
       <concept_desc>Computing methodologies~Reasoning about belief and knowledge</concept_desc>
       <concept_significance>500</concept_significance>
       </concept>
   <concept>
       <concept_id>10010147.10010178.10010216.10010217</concept_id>
       <concept_desc>Computing methodologies~Cognitive science</concept_desc>
       <concept_significance>100</concept_significance>
       </concept>
 </ccs2012>
\end{CCSXML}

\ccsdesc[500]{General and reference~Surveys and overviews}
\ccsdesc[500]{Computing methodologies~Aritificial intelligence}
\ccsdesc[500]{Computing methodologies~Natural language processing}
\ccsdesc[500]{Computing methodologies~Reasoning about belief and knowledge}
\ccsdesc[100]{Computing methodologies~Cognitive science}
%%%%
%%%% Keywords. The author(s) should pick words that accurately describe
%%%% the work being presented. Separate the keywords with commas.
\keywords{AI, NLP, Misinformation, Deepfakes, Opinions, Biases, AI4SI}
%%
\received{16/01/2026}
%%\received[revised]{18/09/2025}
%%\received[accepted]{24/19/2025}

%%
%% This command processes the author and affiliation and title
%% information and builds the first part of the formatted document.
\maketitle

\section{Introduction}

Artificial Intelligence has fascinated public and scientific community alike for almost a century, since Alan Turing's "Imitation Game" \cite{turing_computing_1950}. The idea, trivialized, is that any machine capable of fooling a person into thinking it's speaking to a human can be considered sentient. For seventy-three years the game has remained unbeaten, until OpenAI's ChatGPT-4 win in 2023 \cite{biever_chatgpt_2023}. This model is one of the last iterations of the Generative Pre-Training models pioneered by OpenAI in 2018 (at the moment of writing the latest model available is GPT-5.2) \cite{radford_improving_2018}, which closely followed the first breakthrough towards a human-like agents: "Attention Is All You Need" \cite{vaswani_attention_2017} is a 2017 landmark research paper authored by eight Google researchers that introduced the \textit{transformer} architecture, considered the backbone of all modern LLMs and the main contributor of the AI boom \cite{miller_artificial_2023}.



\section{AI for Fake News Generation and Detection}
\label{sec:ai_fkns}

Fake news have rapidly become a significant concern in the modern digital age, thanks to their virality and potential damages. They spread faster and generate more engagement than truthful information \cite{kleinman_fake_2018,silverman_this_2016}, and can influence public opinion, manipulate elections and pose a threat to public health: the European Union issued guidelines to online platforms and search engines to mitigate the impact on misinformation on elections \cite{baccini_against_2024}, the World Economic forum has identified the proliferation of false content as the leading short-term global risk in 2025 \cite{carson_fake_2025}, and a BBC investigation found Russian-funded fake news networks aiming to disrupt european elections \cite{marocico_how_2025}. Moreover, fake news on health can cause psychological disorders and panic, fear, depression, and fatigue \cite{rocha_impact_2023}, and the World Health Organization called for the development of international fact-checking organizations to combat this phenomenon \cite{world_health_organization_1st_nodate}.  

%% fkns generation

Adding to the problem, the recent advancements in generative artificial intelligence have made it significantly easier to propagate disinformation throughout the web: generated content is increasingly indistinguishable from human-written text, sometimes even perceived as more credible \cite{kreps_all_2020}, citing true evidence to support false claims \cite{feng_syntactic_2012}, and inducing the illusion of majority opinion thanks to the sheer volume of information produced \cite{diresta_ai-generated_2020}.
Some works highlights how

% ai-gen fnks on X 2505.10266

%% deepfakes generation

Artificial agents can generate more than just text: they can create realistic images, videos and sounds, allowing them to reproduce digital twins of real or fictional people, known as deepfakes. In March 2019, such a technology has been used to trick a UK-based energy firm's CEO into transferring $\mathdollar 243{.}000$ to a convincingly mimicked company's German parent firm's CEO \cite{ferrara_genai_2024}. Deepfakes also increased the amount of conspiratorial videos on the internet, and they are especially vicious when targeting children, whose worldviews are easily swayed by deceptive—and highly photorealistic—content \cite{verma_one_2024}.

%% detection techniques

It follows that detecting and mitigating fake news is crucial, especially since the rise of AI-generated content has made disinformation easier to spread and more convincing. From the foundational work by Devlin et al. on \textit{BERT} in 2018 \cite{devlin_bert_2019}, which revolutionized natural language processing trough deep bidirectional transformers, to innovative detection models like \textit{exBAKE} and the application of transformers [TODO]


\section{AI on Humans}
\label{sec:ai_humancentric}

\section{ai own biases and influenceability}
\label{sec:ai_biases}

Artificial agents are trained on a mostly human-generated data, so they learn human biases and tendencies themselves, developing both historical and political preferences in the textual content they generate \cite{roselli_managing_2019,rozado_political_2023,rozado_political_2024}. They tend to exhibit social sycophancy, by agreeing with and flattering the user at the cost of correctness \cite{cheng_elephant_2025}, and they are easily influenceable by small changes in prompt wording \cite{sclar_quantifying_2024, salinas_butterfly_2024}. This can be highly problematic as biased AIs can lead to discrimination or exclusion of marginalized groups, raising ethical concerns \cite{rosario_generative_1}.

%% measure ai bias on us politics: 2503.1064

Rozado, in the context of US politics, measured the political bias of popular large language models \cite{rozado_measuring_2025}. First, they calculated the similarity between AI-generated text and public speeches from Congress representatives (both Democrat and Republican). Then, they used an LLM to annotate as left- or right-leaning AI-generated policy recommendations. Consequently, they did a sentiment analysis on AI-generated comments about american public figures, such as legislators or journalists (figure \ref{fig:rozado_colors}), and lastly they administered three different political orientation tests to the various LLMs. The results show substantial evidence that popular large language models are biased and left-leaning. Table \ref{tab:rozado} shows the three most and three least biased LLMs among those tested.


\begin{figure}[h]
  \centering
  \includegraphics[width=\linewidth]{images/rozando_manycolors.png}
  \caption{Average sentiment (negative: -1, neutral: 0, positive: 1) towards ideologically aligned public figures in conversational LLMs’ generated texts. Statistically significant two-sample t-tests at the 0.01 threshold are indicated with an asterisk. Rozado \cite{rozado_measuring_2025}}
  \Description{Many bar graphs.}
  \label{fig:rozado_colors}
\end{figure}

\begin{table}[h]
    \centering
    \caption{Ranking of political bias in conversational LLMs sorted in ascending
order from least politically biased to most. Rozado \cite{rozado_measuring_2025}}
    \label{tab:rozado}
    \begin{tabular}{lc}
        \toprule
        \textbf{Rank} & \textbf{Model}\\
        \midrule
            1         &   Google Gemma 1.1 2b IT  \\
            2         &   xAI Grok Beta    \\
            3        &   Mistral AI Mistral 7B Instruct v0.2     \\
            ...        &   ...     \\
            18          &   Nous Hermes 2 Mixtral 8x7B DPO     \\
            19     &   Google Gemini 1.5 Pro      \\
            20  &  Google Gemini 1.5 Flash      \\
        \bottomrule
    \end{tabular}
\end{table}

%% influence ai opinions/beliefs: 2510.19107

\section{Ethical AI}
\label{sec:ai_ethical}

\section{conclusion}

This survey highlighted both opportunities and dangers posed by AI technologies, which must be carefully managed. They can be used to tackle challenges previously considered unassailable, such as fake news and hate speech detection, but they can just as much make them worse. They can analyze and mimic human behaviour to increase productivity and aid creativity, but they can also influence people's opinions and reduce their autonomy of judgment.
The consequences they can have upon society are so impactful that they spurred entire research fields towards the developments of ethical and explainable AIs, in the hope of harnessing the power of these technologies toward a better future.
%%
%% The acknowledgments section is defined using the "acks" environment
%% (and NOT an unnumbered section). This ensures the proper
%% identification of the section in the article metadata, and the
%% consistent spelling of the heading.
%% \begin{acks}
%% To Robert, for the bagels and explaining CMYK and color spaces.
%% \end{acks}

%%
%% The next two lines define the bibliography style to be used, and
%% the bibliography file.
\bibliographystyle{ACM-Reference-Format}
\bibliography{report}


\end{document}
\endinput
%%
%% End of file `sample-manuscript.tex'.
