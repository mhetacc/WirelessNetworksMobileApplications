\section{AI on Human Behaviour}
\label{sec:ai_humancentric}

Today's society is already fully dependent on technology: from banking system, to traffic monitoring and public health databases IT systems have become essential. Individuals are in the same situation: virtually everyone in global north under the age of 65 possess and use daily a smartphone \cite{howarth_how_2021}. It follows that artificial intelligence will become an integral part of our personal and professional lives, therefore modeling them to mimic our behaviors could aid in their usefulness and understandability. There are already evidences that humans can exploit them to acquire better comprehension of a phenomenon \cite{schneider_humans_2020}, and they can also enhance creativity in heterogeneous groups \cite{ueshima_simple_2024}. Moreover, LLMs represents a significant methodological shift in computational communication science, enabling a more flexible, more nuanced, but also less controllable exploration of social theories that have historically been difficult to reduce to simple mathematical formalisms \cite{park_generative_2023}.
Overall, AI seems to be a good fit for understanding, modeling and replicating human behaviour.

One possible use of such capabilities is hate speech detection: