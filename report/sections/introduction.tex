\section{Introduction}

Artificial intelligence has fascinated the scientific community for almost a century, spurring famous research papers such as Alan Turing's \textit{"Computing Machinery and Intelligence"} in 1950 \cite{turing_computing_1950}, which introduced the \textit{imitation game}. The idea, trivialized, is that any machine capable of fooling a person into thinking it's speaking to a human can be considered sentient. For seventy-three years the game remained unbeaten, until OpenAI's ChatGPT-4 ultimately succeeded in 2023 \cite{biever_chatgpt_2023}. The model, simulating AGI capabilities \cite{bubeck_sparks_2023}, is one of the last iterations of the Generative Pre-Training LLMs pioneered by OpenAI in 2018 (at the moment of writing the latest available is GPT-5.2) \cite{radford_improving_2018}, which closely followed the first breakthrough towards human-like agents: "Attention Is All You Need" \cite{vaswani_attention_2017} is a 2017 landmark research paper authored by eight Google researchers that introduced the \textit{transformer} architecture, considered the backbone of all modern LLMs and the main contributor of the AI boom \cite{miller_artificial_2023}.

% The general public is not immune to the AI hype either: ChatGPT reached one million users in just five days \cite{mollman_artificial_2022}, which is an astonishing feat when compared to other revolutionary technologies such as personal computers, which needed almost ten years to reach the same milestone \cite{reimer_total_2005}.

Computer scientists are not the only ones engrossed in the topic: philosophers involved themselves too, most notably with Jhon Searle and his 1980s' \textit{chinese room} thought experiment, which directly challenged Turing's ideas and refuted the possibility of true machine intelligence \cite{searle_minds_1980}, while even the general public showed great interest once AIs became useful enough: ChatGPT reached one million users in just five days \cite{mollman_artificial_2022}, which is an astonishing feat when compared to other technologies such as personal computers, which needed almost ten years to reach the same milestone \cite{reimer_total_2005}.
