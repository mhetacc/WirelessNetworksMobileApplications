\section{Ethical AI for positive social impact}
\label{sec:ai_ethical}

Up until this point, this paper highlighted how powerful and dangerous AI technologies can be. Unfortunately, prohibiting their development is not an option: all major global players are heavily investing in this technology. For example, its State Council aims China to be the global leader in the development of artificial intelligence theory and technology by 2030 \cite{webster_chinas_2017}, and Microsoft has announced an investment of thirty billion dollars in the UK on AI infrastructure during the four years from 2025 through 2028 \cite{smith_microsoft_2025}.

It follows that ethical and conscious management of these technologies, to mitigate the societal damages they could do while harnessing their potential, is critical. For this very reason, the White House Office of Science and Technology Policy issued a document "AI for Social Good" (\textit{AI4SI}) in 2016 \cite{computing_community_consortium_artificial_2016}, which pioneered this topic among researchers. 

One example is \textit{"The Social Impact of Generative AI"} by Baldassarre et al. \cite{baldassarre_social_2023}, which evaluates the potential impact on several social sectors and illustrates the findings of a comprehensive literature review of both positive and negative effects, emerging trends, and areas of opportunity of Generative AI models, focusing primarily on ChatGPT. They conclude that two areas are of particular concern, namely \textit{privacy} and \textit{potential biases}.

Moreover, Tambe et al. \cite{tambe_next_2025} examine the historical context and recent surge of AI4SI, highlighting the importance of interdisciplinary collaboration. For example, while training a model for maternal health programs the development team should work with experts in healthcare and social work.

Lastly, Hagerty and Rubinov \cite{hagerty_global_2019} review the literature of recent social science scholarship on the social impacts of artificial intelligence and related technologies in five global regions. Their findings suggest that  AI is likely to have markedly different social impacts depending on geographical setting, and that AI systems have demonstrated a pattern of exacerbating inequality,
often in the most unequal societies and particularly for the most vulnerable populations.

Overall, research on AIs focused on having a positive social impact is still not fully explored, and has the potential of being increasingly more relevant in the future.